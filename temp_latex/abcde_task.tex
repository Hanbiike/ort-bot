\documentclass{stand}
\shownumbering

\begin{document}
\setcounter{comparisoncounter}{5}
\ABCDE[У вида покрытосеменных соматическое хромосомное число равно $2n=12$. В результате нарушения мейоза в образовавшемся женском гаметофите яйцеклетка и оба полярных ядра остались нередуцированными (т.е. каждое из них содержит по $2n$ хромосом). Пыльца нормальная и несёт спермии с набором $n$. В результате двойного оплодотворения один спермий оплодотворяет яйцеклетку, второй — центральную клетку (образующиеся полярные ядра сливаются). Определите хромосомное число зиготы и первичного эндосперма.]{Зигота: $18$, первичный эндосперм: $30$}{Зигота: $18$, первичный эндосперм: $18$}{Зигота: $12$, первичный эндосперм: $30$}{Зигота: $24$, первичный эндосперм: $36$}{Зигота: $18$, первичный эндосперм: $24$}

\end{document}